\documentclass[a5paper]{article}
\usepackage{geometry}
\usepackage{graphicx}

\usepackage{amsmath,amssymb}
\usepackage{physics}
\usepackage{bm}
\usepackage{mathtools}

\usepackage{amsthm}
\newtheorem{definition}{Definition}

\pagenumbering{gobble}

\usepackage{autonum}

\newcommand{\M}{\mathcal{M}}
\newcommand{\N}{\mathcal{N}}
\newcommand{\T}{\mathrm{T}}
\newcommand{\F}{\mathbb{F}}
\newcommand{\R}{\mathbb{R}}
\newcommand{\Cinf}{\mathcal{C}^\infty}

\renewcommand{\eval}[1]{\!\left.{#1}\right|}

\begin{document}

\section*{Pushforward}

If $\psi : \M \to \N$ is a smooth map between manifolds, then the \emph{pushforward} of $\psi$, denoted $\psi_*$ or $\T\psi$ is the induced mapping $\T\M \to \T\N$ between tangent bundles.
Imagine gluing a vector $V \in \T_p\M$ to the manifold $\M$, and continuously moving $\M$ onto $\N$ according to $\psi$, bringing the vector with it.
The final position of the vector is $\psi_*V \in \T_{\psi(p)}\N$.

\begin{center}
	\makebox[\textwidth][c]{
		\includegraphics[width=\textwidth]{{figure.pushforward}.jpg}
	}
\end{center}

\begin{definition}[Pushforward]
	Let $\psi : \M \to \N$ be a smooth map.
	Let $V \in \T_p\M$ be a vector.
	The pushforward $\psi_*V \in \T_{\psi(p)}\N$ of $V$ by $\psi$ is defined by its action on a scalar field $f \in \mathcal C^\infty(\N)$ via
	\begin{gather}
		\psi_* : \T_p\M \to \T_{\psi(p)}\N
	,\\	(\psi_* V)(f)\big|_{\psi(p)} = V(f \circ \psi)\big|_p
	.\end{gather}
\end{definition}
\emph{``The action of $\psi_*V$ on any function is simply the action of $V$ on the pullback of that function.''}
The map $\psi$ need not be invertible.
The pushforward $\psi_*$ may be defined on vector fields $V \subset \T\M$ by applying the definition pointwise for all $p \in \M$.
However, if $V$ is a vector field on $\M$, then $\psi_*V$ is only defined on the image $\psi(\M) \subseteq \N$ and is multi-valued where $\psi$ fails to be injective.

\subsubsection*{Coordinate basis}
To explicitly find $\psi_*$ at a point $\psi(p) \in \N$, let $x^\mu$ and $y^{\bar\mu} = \psi^{\bar\mu}(x^\alpha)$ be local coordinates in a neighbourhood of $p \in \M$ and $\psi(p) \in \N$, respectively.
These coordinates induce bases in the tangent spaces,
\begin{align}
	\T_p\M &= \operatorname{span}\qty{\pdv{x^\mu}}_{\mu=1}^{\dim\M}
,&	\T_{\psi(p)}\N &= \operatorname{span}\qty{\pdv{y^{\bar\mu}}}_{\bar\mu=1}^{\dim\N}
.\end{align}
Unpacking the terse definition,
\begin{align}
	\underbrace{
		(\underbrace{
			\psi_*\underbrace{V}_{\T_p\M}
		}_{\T_{\psi(p)}\N})(\underbrace{f}_{\Cinf(\N)})
	}_\R \Big|_{\psi(p)}
	= \underbrace{
		\underbrace{
			V(\underbrace{f\circ\psi}_{\Cinf(\N)})
		}_{\Cinf(\N)}\Big|_p
	}_\R
	\label{eqn:def-for-functional}
,\end{align}
we proceed to write the left-hand side in coordinates,
\begin{align}
	\eval{(\psi_*V)(f)}_{\psi(p)}
	&= (\psi_*V)^{\bar\mu}
		\eval{\pdv{f}{y^{\bar\mu}}}_{\psi(p)}
	= \eval{\qty(\psi_*(V^\alpha\partial_\alpha))^{\bar\mu}}_{\psi(p)}
	\eval{\pdv{f}{y^{\bar\mu}}}_{\psi(p)}
.\end{align}
Equating with the right-hand side, also expressed in coordinates,
\begin{align}
	\eval{V(f\circ\psi)}_p
	&= V^\nu
		\eval{\pdv{f(\psi^{\bar\alpha}(x^\beta))}{x^\nu}}_p
	= V^\nu
		\eval{\pdv{f}{y^{\bar\mu}}}_{\psi(p)}
		\eval{\pdv{\psi^{\bar\mu}}{x^\nu}}_p
,\end{align}
we cancel $\eval{\pdv*{f}{y^{\bar\mu}}}_{\psi(p)}$ from both sides and obtain
\begin{align}
	\eval{\qty(\psi_*(V^\alpha\partial_\alpha))^{\bar\mu}}_{\psi(p) }
	= V^\nu\eval{\pdv{y^{\bar\mu}}{x^\nu}}_p
.\end{align}
From this we see that $\psi_*$ is linear, allowing expression as a matrix.
% \begin{math}
% 	(\psi_*(V))^{\bar\mu} = (\psi_*)^{\bar\mu}{}_\nu V^\nu
% .\end{math}
\begin{align}
	\eval{(\psi_*)^{\bar\mu}{}_\nu}_{\psi(p)} V^\nu
	= V^\nu \eval{\pdv{y^{\bar\mu}}{x^\nu}}_p
\end{align}
Finally, since $V^\mu$ is arbitrary, we may peel it off, leaving
\begin{align}
	\eval{(\psi_*)^{\bar\mu}{}_\nu}_{\psi(p)} = \eval{\pdv{y^{\bar\mu}}{x^\nu}}_p
	\iff
	\psi_* = \eval{\pdv{y^{\bar\mu}}{x^\nu}}_p \partial_{\bar\mu} \otimes \dd x^\nu
.\end{align}


\subsubsection*{Relation to Exterior Derivative}


In the case that the codomain $\N \cong \R$ is one-dimensional, the pushforward of vectors in $\T\M$ by $\psi$ coincides with the exterior derivative $\dd\psi$ of $\psi$ when viewed as a scalar field in $\Cinf(\M)$.

In coordinates, $(\psi_*)^{\bar\mu}{}_\nu = \pdv*{\psi^{\bar\mu}}{x^\nu}$, but $\bar\mu$ runs over a single coordinate since $\dim\R = 1$, so the index may be implicitly dropped.
\begin{align}
	(\psi_*)_\mu = \pdv{\psi}{x^\mu}
	\iff
	\psi_* = \pdv{\psi}{x^\mu}\dd x^\mu \equiv \dd\psi
.\end{align}
Formally, the pushforward is a linear map $\psi_* : \T\M \to \T\R$, whereas the exterior derivative is a linear map $\dd\psi : \T\M \to \R$.
If $\xi$ is the single coordinate of $\R$, so that $\T\R = \operatorname{span}\qty{\partial_\xi}$ and $\T^*\R = \operatorname{span}\qty{\dd\xi}$, then the exterior derivative may be defined in terms of the pushforward by
\begin{align}
	\dd\psi(V) \coloneqq \psi_*(\dd\xi \otimes V)
	&= \pdv{\xi}{x^\nu} \braket{\partial_\xi}{\dd\xi} \braket{\dd x^\nu}{V^\mu\partial_\mu}
\\	&= V^\mu\pdv{\xi}{x^\nu} \delta^\nu_\mu
	= V^\mu\pdv{\xi}{x^\mu}
.\end{align}
%
%
\end{document}
